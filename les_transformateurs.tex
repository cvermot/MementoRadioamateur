    Lors de l'examen, il ne sera considéré que le cas des transformateurs dits parfaits, c'est à dire dont le rendement est égale à 100\% Dans la réalité, ce genre de transformateur n'existe pas, des pertes sont toujours présentes, et elles se manifestent par un échauffement du transformateur (perte par effet Joule : toute l'énergie transformée en chaleur est "perdue").
    
    Le transformateur est représenté dans les circuits par le schéma suivant :
    \begin{center}
    \shorthandoff{:!}
   \begin{circuitikz} \draw
(0,0) node[transformer] (T) {}
(T.A1) node[anchor=east] {P1}
(T.A2) node[anchor=east] {P2}
(T.B1) node[anchor=west] {S1}
(T.B2) node[anchor=west] {S2}
(T.base) node[anchor=south]{Transformateur parfait}
;\end{circuitikz}
 \end{center}
 
    On distingue 2 "circuits" sur un transformateur : le primaire et le secondaire. On branche l'alimentation sur le primaire (donc sur le schéma ci-dessus, on branche l'alimentation sur les bornes P1 et P2 (connexion au primaire) et on récupère le courant sur le secondaire (donc entre les bornes S1 et S2).
    
    On peut constater que la représentation du transformateur fait figurer 2 bobines. Cela est assez proche de la réalité, car un transformateur est effectivement composé de 2 enroulements : le premier enroulement est le primaire, et le second le secondaire. Il est intéressant de noter qu'il n'y a pas de connexion "électrique" entre ces 2 enroulements, qui sont (et doivent êtres) parfaitement isolés électriquement. Le transfert d'énergie entre les 2 enroulements se fait donc uniquement par induction. Hors, tout comme dans une bobine, il faut pour qu'il y ait une induction que le courant au primaire soit alternatif : \textbf{un transformateur ne fonctionne donc qu'avec du courant alternatif !!!}
    
    Donc le branchement de la figure 1 fournira une tension dans la résistance R, en revanche, aucune tension n'existera dans le secondaire de la figure 2 car une pile produit un courant continu.
    
        \begin{center}
        \shorthandoff{:!}
   \begin{circuitikz} \draw
(0,0) node[transformer] (T) {}

(-2,-2.1) to[sI=$A$] (-2,0)
(-2,-2.1) -- (T.A2)
(-2,0) -- (T.A1)

(2,-2.1) to[european resistor, l_=$R$] (2,0)
(2,-2.1) -- (T.B2)
(2,0) -- (T.B1)
;\end{circuitikz}\begin{circuitikz} \draw
(0,0) node[transformer] (T) {}

(-2,-2.1) to[battery=$P$] (-2,0)
(-2,-2.1) -- (T.A2)
(-2,0) -- (T.A1)

(2,-2.1) to[european resistor, l_=$R$] (2,0)
(2,-2.1) -- (T.B2)
(2,0) -- (T.B1)
;\end{circuitikz}
 \end{center}
 
    \subsection{Rapport de tension}
    Nous n'avons pour le moment pas parlé du rapport de transformation, pourtant, c'est bien la le rôle d'un transformateur : transformer une tension. Le plus souvent, on à affaire à des transformateurs réducteur de tension, mais il faut savoir qu'en branchant un transformateur réducteur de tension à l'envers, on obtient un élévateur de tension.
    
    Ce qui détermine la tension de sortie d'un transformateur, c'est tout simplement le rapport entre le nombre de spires au secondaire (que nous allons noter $N_S$) et le nombre de spires au primaire (noté $N_P$).
    
    \begin{center}
    \shorthandoff{:!}
   \begin{circuitikz} \draw
(0,0) node[transformer] (T) {}

(-2,-2.1) to[sI=$220V$] (-2,0)
(-2,-2.1) -- (T.A2)
(-2,0) -- (T.A1)

(2,-2.1) to[european resistor, l_=$R$] (2,0)
(2,-2.1) -- (T.B2)
(2,0) -- (T.B1)
(T.B2) node[anchor=north] {10 spires}
(T.A2) node[anchor=north] {20 spires}

(4,-1.85) to[voltmeter, l_=$110V$] (4,-0.25)
(4,-1.85) -- (2,-1.85)
(4,-0.25) -- (2,-0.25)

;\end{circuitikz}
 \end{center}
 
    Dans l'exemple ci-dessus, on constate qu'on a 20 spires au primaire ($N_P = 20$) et 10 spires au primaire ($N_S = 10$). On calcul donc le rapport : $r = \dfrac{N_S}{N_P} = \dfrac{10}{20} = 0,5$. Le rapport de transformation de ce transformateur est donc de 0,5, donc la tension de sortie $T_S$ est égale : $T_S = T_E \times r = 220 \times 0,5 = 110V$. \\

	Une autre méthode de calcul est celle des "volts par spire". Considérant que nous avons un primaire de 20 spires alimenté en 220V, on pose $\dfrac{220}{20} = 11$ volts/spire. On a 10 spires au secondaire, donc $11 \times 10 = 110 V$.
    
    \subsection{Puissance et transformateur}
    On a vu dans l'introduction sur les transformateurs que durant l'examen, on ne considérera que le cas des transformateurs parfaits, c'est à dire le cas où aucune perte ne se produit (pas d'échauffement). Il faut donc garder en tête que dans ce cas, \textbf{la puissance consommée en entrée est égale à la puissance délivrée en sortie}. Cela permet de résoudre des problèmes du type : dans le montage suivant, quelle est l'intensité au secondaire ?

 \begin{center}
 \shorthandoff{:!}
   \begin{circuitikz} \draw
(0,0) node[transformer] (T) {}
(T.base) node[anchor=south]{Transformateur parfait}

(-2,-2.1) to[sI=$300W$] (-2,0)
(-2,-2.1) -- (T.A2)
(-2,0) -- (T.A1)

(2,-2.1) to[european resistor, l_=$75V$] (2,0)
(2,-2.1) -- (T.B2)
(2,0) -- (T.B1)
;\end{circuitikz}
 \end{center}
 
 On sait donc que si on a 300W au primaire, on a également 300W au secondaire. Il suffit ensuite d'appliquer la formule de la puissance ($P = U \times I)$, et on obtient rapidement que le courant dans le secondaire équivaut à 4A.
